
\documentclass{book}
%%%%%%%%%%%%%%%%%%%%%%%%%%%%%%%%%%%%%%%%%%%%%%%%%%  
%%%%%%%%%%%%%%%%% BEGIN PREAMBLE %%%%%%%%%%%%%%%%%
%%%%%%%%%%%%%%%%%%%%%%%%%%%%%%%%%%%%%%%%%%%%%%%%%%  

%%%%%%%%%%%%%%%%%%%% PACKAGES %%%%%%%%%%%%%%%%%%%%
\usepackage{amsmath}
\usepackage{amsthm}
\usepackage[margin=0.65in]{geometry}
\usepackage[parfill]{parskip}
\usepackage{tcolorbox}
\tcbuselibrary{theorems}
\tcbuselibrary{skins}
\tcbuselibrary{breakable}
\usepackage{cleveref}
\usepackage{lipsum}  %just used to generate lorem ipsum text

%%%%%%%%%%%%%%%%% CUSTOM COLORS %%%%%%%%%%%%%%%%%%  
%% colorblind friendly pallet using VT colors
\definecolor{cbred}{HTML}{861F41} %VT color
\definecolor{cborange}{HTML}{E87722} %VT color
\definecolor{cbyellow}{HTML}{F7EA48} %VT color
\definecolor{cbgreen}{HTML}{1B7D32} %not VT color
\definecolor{cbblue}{HTML}{003C71} %VT color
\definecolor{cbgray}{HTML}{D7D2CB} %VT color

%%%%%%%%%%%%%%%%% THEOREM STYLES %%%%%%%%%%%%%%%%%  
%% fancy using tcolorbox
\newtcbtheorem[number within=section]{thm}{Theorem}%
{breakable,enhanced,colback=cbblue!5!white,colframe=cbblue!50!black,fonttitle=\bfseries,label type=theorem}{thm}
\crefname{theorem}{Theorem}{Theorems}
\newtcbtheorem[number within=subsection, use counter from =thm]{prop}{Proposition}%
{breakable,enhanced,colback=cbblue!5!white,colframe=cbblue!50!black,fonttitle=\bfseries,label type=proposition}{thm}
\crefname{proposition}{Proposition}{Propositions}
\newtcbtheorem[number within=subsection, use counter from =thm]{lem}{Lemma}%
{breakable,enhanced,colback=cbblue!5!white,colframe=cbblue!50!black,fonttitle=\bfseries,label type=lemma}{thm}
\crefname{lemma}{Lemma}{Lemmas}
\newtcbtheorem[number within=subsection, use counter from =thm]{cor}{Corollary}%
{breakable,enhanced,colback=cbblue!5!white,colframe=cbblue!50!black,fonttitle=\bfseries,label type=corollary}{thm}
\crefname{corollary}{Corollary}{Corollaries}
\newtcbtheorem[number within=subsection, use counter from =thm]{defn}{Definition}%
{breakable,enhanced,colback=cbgreen!5!white,colframe=cbgreen!50!black,fonttitle=\bfseries,label type=definition}{defn}
\crefname{definition}{Definition}{Definitions}
\newtcbtheorem[number within=subsection, use counter from =thm]{notdefn}{Notation}%
{breakable,enhanced,colback=cbgreen!5!white,colframe=cbgreen!50!black,fonttitle=\bfseries,label type=definition}{notdefn}
\crefname{definition}{Notation}{Notation}
\newtcbtheorem[number within=subsection, use counter from =thm]{ex}{Example}%
{breakable,bicolor,coltitle=black,colback=cborange!25!white,colbacklower=white,colframe=cborange,fonttitle=\bfseries,label type=example}{ex}
\crefname{example}{Example}{Examples}
\newtcbtheorem[number within=subsection, use counter from =thm]{exc}{Exercise}%
{breakable,enhanced,colback=cbred!5!white,colframe=cbred!50!black,fonttitle=\bfseries,label type=exercise}{exc}
\crefname{exercise}{Exercise}{Exercises}
\newtcbtheorem[number within=subsection, use counter from =thm]{alg}{Algorithm}%
{breakable,enhanced,colback=gray!5!white,colframe=gray!50!black,fonttitle=\bfseries,label type=algorithm}{alg}
\crefname{algorithm}{Algorithm}{Algorithms}
%% fancy proof
\tcolorboxenvironment{proof}{% modifies `proof' from `amsthm'
blanker,breakable,left=5mm,before skip=10pt,after skip=10pt,
borderline west={1mm}{0pt}{cbblue!50!black}}
%% unnumbered normal theorem style
\newtheorem*{claim}{Claim}
%% unnumbered conjectures and questions 
\theoremstyle{definition}
\newtheorem*{conj}{Conjecture}
\newtheorem*{ques}{Question}
%% remarks, facts, italics (no bold) titles
\theoremstyle{remark}
\newtheorem*{rmk}{Remark}
\newtheorem*{fact}{Fact}

%%%%%%%%%%%%%%%%%%%%%%%%%%%%%%%%%%%%%%%%%%%%%%%%%%  
%%%%%%%%%%%%%%%%%% END PREAMBLE %%%%%%%%%%%%%%%%%%
%%%%%%%%%%%%%%%%%%%%%%%%%%%%%%%%%%%%%%%%%%%%%%%%%%  

\title{Notes Layout}
\author{Joe Wells}
\date{April 2024}

\begin{document}
\raggedright

\frontmatter

\maketitle

\newpage

\tableofcontents

\newpage

\chapter{Preface}

\lipsum[1-2]

\newpage

\mainmatter

\chapter{chapter title}

You can easily link theorems internally.
\begin{itemize}
\item \Cref{thm:fund-thm-of-thms}
\item \Cref{thm:shorthand-name-for-prop}
\item \Cref{thm:helper}
\item \Cref{thm:totes-obvs}
\item \Cref{defn:words-to-live-by}
\item \Cref{ex:importante}
\item \Cref{exc:practice-practice}
\end{itemize}

\section{section title}

\begin{thm}{Fundamental Theorem of Theorems}{fund-thm-of-thms}
  \lipsum[1-1]
\end{thm}

\begin{proof}
  \lipsum[2-3]
\end{proof}

\begin{prop}{An interesting result}{shorthand-name-for-prop}
  \lipsum[4-4]
\end{prop}

\begin{lem}{This Helps Prove Theorems}{helper}
  \lipsum[5-5]
\end{lem}

\begin{cor}{Obviously}{totes-obvs}
  \lipsum[6-6]
\end{cor}

\subsection{subsection title}

\lipsum[7-7]

\begin{defn}{Important Term}{words-to-live-by}
  \lipsum[8-8]
\end{defn}

\begin{defn*}{Less Important Term} %asterisk removes numbering of definition
  \lipsum[8-8]
\end{defn*}

\begin{ex}{Important Example}{importante}
  \lipsum[9-9]
  \tcblower
  \lipsum[10-10]
\end{ex}

\begin{exc}{Now you}{practice-practice}
  \lipsum[11-11]
\end{exc}

\section{next section title}

\lipsum[12-12]

\begin{claim}
  \lipsum[13-13]
\end{claim}

\begin{conj}
  \lipsum[14-14]
\end{conj}

\begin{ques}
  \lipsum[15-15]
\end{ques}

\begin{rmk}
  \lipsum[16-16]
\end{rmk}

\begin{fact}
  \lipsum[17-17]
\end{fact}

\chapter{next chapter title}

\lipsum[18-18]

\section{next section title}

\lipsum[19-19]

\section{next subsection title}

\lipsum[20-20]

\end{document}
