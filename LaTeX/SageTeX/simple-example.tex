\documentclass[12pt]{article}
\usepackage{amsfonts,amsmath,amsthm,amssymb}
\usepackage{sagetex}

\begin{document}

%The following creates a matrix B which is in SL(3,Z) by performing elementary row operations to the identity matrix.
\begin{sagesilent}
  latex.matrix_delimiters('[', ']');
  latex.matrix_column_alignment('c');
  A=identity_matrix(3);
  A.add_multiple_of_row(0,1,randint(2,4));
  A.add_multiple_of_row(1,2,randint(-3,-1));
  A.add_multiple_of_row(2,0,randint(1,2));
\end{sagesilent}

\textbf{Problem 1.} Invert the matrix $$ A = \sage{A}.$$

\textbf{Solution 1.} $$A^{-1} = \sage{A.inverse()}$$

\dotfill

%The following creates a matrix B which is in SL(3,Z) by performing elementary row operations to the identity matrix.
\begin{sagesilent}
  latex.matrix_delimiters('[', ']');
  latex.matrix_column_alignment('c');
  B=identity_matrix(3);
  B.add_multiple_of_row(0,1,randint(2,4));
  B.add_multiple_of_row(1,2,randint(-3,-1));
  B.add_multiple_of_row(2,0,randint(1,2));
\end{sagesilent}

\textbf{Problem 2.} Invert the matrix $$B = \sage{B}.$$

\textbf{Solution 2.} $$B^{-1} = \sage{B.inverse()}$$

\dotfill

%The following creates a matrix C which is in SL(3,Z) by performing elementary row operations to the identity matrix.
\begin{sagesilent}
  latex.matrix_delimiters('[', ']');
  latex.matrix_column_alignment('c');
  C=identity_matrix(3);
  C.add_multiple_of_row(0,1,randint(2,4));
  C.add_multiple_of_row(1,2,randint(-3,-1));
  C.add_multiple_of_row(2,0,randint(1,2));
\end{sagesilent}

\textbf{Problem 3.} Invert the matrix $$C = \sage{C}.$$

\textbf{Solution 3.} $$C^{-1} = \sage{C.inverse()}$$


\end{document}
